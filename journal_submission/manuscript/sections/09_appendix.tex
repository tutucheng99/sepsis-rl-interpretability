% ============================================================
% APPENDIX
% ============================================================

\appendix

\section{LEG Interpretability Analysis Details}\label{appendix:leg}

\subsection{LEG Method Formulation}

LEG approximates $\nabla_s Q(s_0, \pi(s_0))$ via: (1) Sample perturbations $Z_i \sim \mathcal{N}(0, \sigma^2 I)$; (2) Compute Q-value differences $y_i = Q(s_i, \pi(s_i)) - Q(s_0, \pi(s_0))$ for $s_i = s_0 + Z_i$; (3) Ridge regression: $\hat{\gamma} = (\Sigma + \lambda I)^{-1}(\frac{1}{n} \sum_i y_i Z_i)$ where $\Sigma = \frac{1}{n} \sum_i Z_i Z_i^\top$. Saliency for feature $j$ is $\hat{\gamma}_j$.

\subsection{Implementation Details}

LEG applied to all algorithms using unified implementation. For Q-learning methods, LEG directly perturbs states; for BC, we use pseudo-Q-value $Q_{\text{BC}}(s, a) = \log \pi(a|s)$. Parameters: $n=1000$ perturbations, $\sigma=0.1$, $\lambda=10^{-6}$, 10 representative states per algorithm, excluding categorical features.

\subsection{Interpretability Metrics}

Three interpretability metrics: \textbf{maximum saliency} ($\max_j |\hat{\gamma}_j|$, signal strength), \textbf{saliency range} ($\max_j \hat{\gamma}_j - \min_j \hat{\gamma}_j$, feature differentiation), and \textbf{clinical coherence} (alignment with sepsis treatment knowledge).
